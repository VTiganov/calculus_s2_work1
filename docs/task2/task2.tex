\documentclass[a4paper,12pt]{article}
\usepackage[T2A]{fontenc}
\usepackage[utf8]{inputenc}
\usepackage[russian]{babel}
\usepackage{graphicx}
\usepackage{float}
\usepackage{subcaption}
\usepackage{amsmath, amssymb}
\usepackage{geometry}
\geometry{top=2cm, bottom=2cm, left=3cm, right=1.5cm}

\begin{document}

\thispagestyle{empty}
\begin{center}
    \large
    Министерство науки и высшего образования Российской Федерации\\
    Федеральное государственное автономное образовательное учреждение\\
    высшего образования\\
    «Национальный исследовательский университет ИТМО»\\
    \vspace{5cm}
    \textbf{Отчёт по исследовательской работе № 1}\\
    \textbf{По предмету: Математический анализ и основы вычислений}\\
    \vspace{6cm}
    \begin{flushright}
        Выполнил работу:\\ Тиганов Вадим Игоревич\\
        \vspace{1cm}
        Академическая группа: \\ J3112\\
        \vspace{1cm}
        Вариант: \\18
    \end{flushright}
    \vspace{1cm}
    \vspace{3cm}
    \begin{center}
        Санкт-Петербург, 2025\\
    \end{center}
\end{center}

\newpage


\section{Ход работы}


\subsection{Задание 2}

Требуется:
\begin{enumerate}
    \item Преобразовать выражение к интегральной сумме,
    \item Доказать существование соответствующего интеграла,
    \item Найти предел:
\end{enumerate}

\[
\lim_{n \to \infty} \sum_{k=1 - n}^{n} \frac{k}{kn + 2n^2}
\]

\subsection*{Графическая интерпретация интеграла}

\begin{figure}[H]
    \centering
    \includegraphics[width=0.9\linewidth]{../img/integral_plot.png}
    \caption{График функции $f(x) = \frac{x}{x+2}$ с выделенной областью интегрирования $[-1,1]$}
    \label{fig:integral}
\end{figure}

\emph{Решение задачи:}

\begin{figure}[H]
    \centering
    \includegraphics[width=0.8\linewidth]{../img/2_1.jpg}
    \caption{}
    \label{fig:part1}
\end{figure}

\begin{figure}[H]
    \centering
    \includegraphics[width=0.8\linewidth]{../img/2_2.jpg}
    \caption{}
    \label{fig:part2}
\end{figure}

Таким образом, легко выделилось количество промежутков интегрирования, сама функция от $x$, рассмотрел края и пришел к формуле Ньютона-Лейбница, вычислив определенный интеграл.\\
\emph{Задача решена}

\end{document}